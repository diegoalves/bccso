\documentclass[a4paper,12pt]{article}

\usepackage[brazil]{babel}
\usepackage[latin1]{inputenc}

\title{Relatorio do Algoritmo de PI}
\author{Diego Alves\\Centro Universitário Senac}
\date{Abril de 2011}

\begin{document}

    \maketitle

    \section{Introdução}

        A partir do que foi proposto em sala de aula, implementar um algoritmo que encontre o número de pi através de números aleatórios, e fazer processamento paralelo.

    \section{Conclusões}
	
        Foi implementado usando a função rand para escolher numeros aleatoriamente, depois usando a função quadrática foi adicionando os numeros que cairia na área limitada para se aproximar da função que gera o número de PI.

\end{document}
